\documentclass[]{beamer}%[handout]
\usepackage{pgfpages}
\setbeamercolor{item projected}{fg=black,bg=white}
\usetheme{Frankfurt}%Warsaw Madrid Frankfurt
\usecolortheme{seahorse}%dove seagull seahorse 
\setbeamertemplate{mini frames}{}	%removes dots in header but creates navigation bar
\beamertemplatenavigationsymbolsempty	% removes navigation bar
\defbeamertemplate{footline}{author and page number}{%
    \usebeamercolor[fg]{page number in head/foot}%
    \usebeamerfont{page number in head/foot}%
    \hspace{1em}\insertshortauthor\hfill%
    \insertpagenumber\,/\,\insertpresentationendpage\kern1em\vskip2pt%\insertframenumber\,/\,\inserttotalframenumber\kern1em\vskip2pt
    }
\setbeamertemplate{footline}[author and page number]{}

\usefonttheme[onlymath]{serif}
\usepackage[utf8]{inputenc}

\usepackage[ngerman]{babel}
\usepackage{graphicx}
\graphicspath{{./pictures/}{./zeichnungen/}}
\usepackage{longtable}
\usepackage{latexsym}
\usepackage{amsmath}
\usepackage{mathtools}
\usepackage{amsfonts}
\usepackage{amssymb}
\usepackage[square,authoryear]{natbib}%square oder round
\usepackage{palatino}
\usepackage{xcolor}

\usepackage[per-mode=symbol,decimalsymbol=comma]{siunitx}
\usepackage{caption}
\captionsetup[figure]{labelformat=empty} %Remove "Abbildung"

\usepackage{import}
\usepackage{multicol}
\usepackage{epsfig}
\usepackage{float}
\usepackage[]{hyperref}

\renewcommand{\ln}[1]{{\rm ln}\left(#1\right)}
\usepackage[absolute,overlay]{textpos}
\setlength{\TPHorizModule}{1mm}
\setlength{\TPVertModule}{\TPHorizModule}

\usepackage{appendixnumberbeamer}

\usepackage{tikz}
\usetikzlibrary{math,fit,arrows,shapes,shapes.misc,positioning,patterns}
\usepackage{environ}
\usepackage{tikz-3dplot} %requires 3dplot.sty to be in same directory, or in your LaTeX installation
\makeatletter
\newsavebox{\measure@tikzpicture}
\NewEnviron{scaletikzpicturetowidth}[1]{%
  \def\tikz@width{#1}%
  \def\tikzscale{1}\begin{lrbox}{\measure@tikzpicture}%
  \BODY
  \end{lrbox}%
  \pgfmathparse{#1/\wd\measure@tikzpicture}%
  \edef\tikzscale{\pgfmathresult}%
  \BODY
  }
\makeatother

\usepackage{listings}

\tikzset{cross/.style={cross out, draw=black, minimum size=2*(#1-\pgflinewidth), inner sep=0pt, outer sep=0pt},
%default radius will be 1pt. 
cross/.default={1pt}}


\definecolor{keywords}{RGB}{0,0,0}
\definecolor{comments}{RGB}{0,0,113}
\definecolor{red}{RGB}{160,0,0}
\definecolor{green}{RGB}{0,150,0}

% Boehlke
% SCHRIFTGROESSE
\newcommand{\Ds}{\displaystyle}
\newcommand{\Ts}{\textstyle}
\newcommand{\Ss}{\scriptstyle}%
% OPERATOREN
\renewcommand{\d}{{\,\rm  d}}
\newcommand{\D}{{\,\rm  D}}
\newcommand{\dt}{\triangle t}
\newcommand{\T}[1]{{#1}^{\sf  T}}
\newcommand{\I}[1]{{#1}^{\rm -1}}
%\newcommand{ \I}[1]{{#1}^{{\mbox{\rule[0.7mm]{0.9mm}{0.2mm}}\sf 1}}}
\newcommand{\IT}[1]{{#1}^{\sf  -T}}
%\newcommand{\IT}[1]{{#1}^{{\mbox{\rule[0.7mm]{0.9mm}{0.2mm}}\sf T}}}
\newcommand{\pd}[2]{\displaystyle\frac{\partial #1}{\partial #2}}
\newcommand{\td}[2]{\frac{{\rm d} #1}{{\rm d} #2}}
\newcommand{\pdn}[3]{\displaystyle\frac{\partial^{#3} #1}{\partial #2^{#3}}}
\newcommand{\tdn}[3]{\frac{{\rm d}^{#3} #1}{{\rm d} #2^{#3}}}
\newcommand{\pdf}[2]{\displaystyle{\partial #1}{/\partial #2}}
\newcommand{\tdf}[2]{\displaystyle{\d #1}{/\d #2}}
\newcommand{\tdfn}[3]{\displaystyle{\d^{#3} #1}{/\d #2^{#3}}}
\newcommand{\pdfn}[3]{\displaystyle{\partial^{#3} #1}{/\partial #2^{#3}}}
\newcommand{\z}[1]{{\tilde{#1}}}
\newcommand{\abs}[1]{|#1|}
\newcommand{\absg}[1]{\left| #1 \right|}
\newcommand{\norm}[1]{\Vert #1 \Vert}
\newcommand{\normg}[1]{\left\Vert \, #1 \, \right\Vert}
\newcommand{\sign}[1]{{\rm sgn}\left( #1 \right)}
\newcommand{\grad}[1]{{\rm grad}\left( #1 \right)}
\renewcommand{\div}[1]{{\rm div }\left( #1 \right)}
%\newcommand{\rot}[1]{{\rm rot }\left( #1 \right)}
\newcommand{\rot}[1]{{\rm curl }\left( #1 \right)}
\newcommand{\Grad}[1]{{\rm Grad}\left( #1 \right)}
\newcommand{\Div}[1]{{\rm Div }\left( #1 \right)}
\newcommand{\Rot}[1]{{\rm Curl }\left( #1 \right)}
%\newcommand{\Rot}[1]{{\rm Rot }\left( #1 \right)}
\newcommand{\tr}[1]{{\rm tr  }( #1 )}
\newcommand{\cof}[1]{{\rm cof  }( #1 )}
\renewcommand{\det}[1]{{\rm det }( #1 )}
\newcommand{\dev}[1]{{\rm dev}( #1 )}
\newcommand{\fsym}[1]{{\rm sym }( #1 )}
\newcommand{\fskw}[1]{{\rm skw }( #1 )}
\newcommand{\unit}[1]{~\rm #1}
\newcommand{\lb}{\left(}
\newcommand{\rb}{\right)}
\newcommand{\la}{\langle}
\newcommand{\ra}{\rangle}
\newcommand{\ls}{\{}
\newcommand{\rs}{\}}
\newcommand{\jl}{\llbracket} %Sprungklammern
\newcommand{\jr}{\rrbracket} %Sprungklammern
% Farben
%\newcommand{\cred}{}
%\newcommand{\cblue}{}
%\newcommand{\cgreen}{}
\newcommand{\cred }{\color{red}}
\newcommand{\cblue }{\color{blue}}
\newcommand{\cgreen }{\color{green}}
%MARKER
\newcommand{\ma}{$\bigstar$}
\newcommand{\mb}{$\blacksquare$}

% ABSCHNITT-STYLES
\newcommand{\fliterature}[1]{{\tiny #1}}
\newcommand{\f}[1]{\mbox{$ #1 $}}
%\newcommand{\ffbox}[1]{\begin{displaymath}\mbox{\fboxsep=.1in \framebox{$ #1 $}}\end{displaymath}}
\newcommand{\eqnbox}[2]{\begin{equation}\mbox{\fboxsep=#1 \framebox{$\Ds #2 $}}\end{equation}}
\newcommand{\topic}[1]{\vspace*{2mm}{\bf  #1.}}
\newcommand{\supplement}[2]{{\scriptsize $\blacktriangledown$ {\bf Ergänzung: #1.} #2}}
%%%\newcommand{\example}[2]{{\scriptsize $\blacktriangleright$ {\bf Beispiel: #1.} #2}}
%
% INDIZES
\newcommand{\ta}{{}^n}
\newcommand{\tb}{{}^{n+1}}
\newcommand{\tba}{{}^{n+1}_i}
\newcommand{\tbb}{{}^{n+1}_{i+1}}
\renewcommand{\r}[2]{{#2_{\langle #1 \rangle}}}
%
%
% ROEMISCHE ZAHLEN
\newcommand{\rI}{{\mathit{I}}}
\newcommand{\rII}{{\mathit{II}}}
\newcommand{\rIII}{{\mathit{III}}}
% FETTE BUCHSTABEN
%\newcommand{\st y}[1]{{\bf #1}}
\newcommand{\sty}[1]{\mbox{\boldmath $#1$}}
%\newcommand{\styy}[1]{\mbox{\boldmath $\sf #1$}}
\newcommand{\styy}[1]{{\mathbb{#1}}}

\newcommand{\fa}{\sty{ a}}
\newcommand{\fb}{\sty{ b}}
\newcommand{\fc}{\sty{ c}}
\newcommand{\fd}{\sty{ d}}
\newcommand{\fe}{\sty{ e}}
\newcommand{\ff}{\sty{ f}}
\newcommand{\fg}{\sty{ g}}
\newcommand{\fh}{\sty{ h}}
%\renewcommand{\fi}{\sty{ i}}
\newcommand{\fj}{\sty{ j}}
\newcommand{\fk}{\sty{ k}}
\newcommand{\fl}{\sty{ l}}
\newcommand{\fm}{\sty{ m}}
\newcommand{\fn}{\sty{ n}}
\newcommand{\fo}{\sty{ o}}
\newcommand{\fp}{\sty{ p}}
\newcommand{\fq}{\sty{ q}}
\newcommand{\fr}{\sty{ r}}
\newcommand{\fs}{\sty{ s}}
\newcommand{\ft}{\sty{ t}}
\newcommand{\fu}{\sty{ u}}
\newcommand{\fv}{\sty{ v}}
\newcommand{\fw}{\sty{ w}}
\newcommand{\fx}{\sty{ x}}
\newcommand{\fy}{\sty{ y}}
\newcommand{\fz}{\sty{ z}}
%
\newcommand{\cfu}{\check{\fu}}
\newcommand{\cfv}{\check{\fv}}
\newcommand{\cfp}{\check{\fp}}
\newcommand{\cfD}{\check{\fD}}
\newcommand{\cfeps}{\check{\feps}}
\newcommand{\cftau}{\check{\ftau}}
\newcommand{\cfsigma}{\check{\fsigma}}
%
\newcommand{\hfu}{\hat{\fu}}
\newcommand{\hfeps}{\hat{\feps}}
\newcommand{\hfsigma}{\hat{\fsigma}}
%
\newcommand{\ffa}{\styy{ a}}
\newcommand{\ffb}{\styy{ b}}
\newcommand{\ffc}{\styy{ c}}
\newcommand{\ffd}{\styy{ d}}
\newcommand{\ffe}{\styy{ e}}
\newcommand{\fff}{\styy{ f}}
\newcommand{\ffg}{\styy{ g}}
\newcommand{\ffh}{\styy{ h}}
%\renewcommand{\fi}{\styy{ i}}
\newcommand{\ffj}{\styy{ j}}
\newcommand{\ffk}{\styy{ k}}
\newcommand{\ffl}{\styy{ l}}
\newcommand{\ffm}{\styy{ m}}
\newcommand{\ffn}{\styy{ n}}
\newcommand{\ffo}{\styy{ o}}
\newcommand{\ffp}{\styy{ p}}
\newcommand{\ffq}{\styy{ q}}
\newcommand{\ffr}{\styy{ r}}
\newcommand{\ffs}{\styy{ s}}
\newcommand{\fft}{\styy{ t}}
\newcommand{\ffu}{\styy{ u}}
\newcommand{\ffv}{\styy{ v}}
\newcommand{\ffw}{\styy{ w}}
\newcommand{\ffx}{\styy{ x}}
\newcommand{\ffy}{\styy{ y}}
\newcommand{\ffz}{\styy{ z}}
%
\newcommand{\fA}{\sty{ A}}
\newcommand{\fB}{\sty{ B}}
\newcommand{\fC}{\sty{ C}}
\newcommand{\fD}{\sty{ D}}
\newcommand{\fE}{\sty{ E}}
\newcommand{\fF}{\sty{ F}}
\newcommand{\zfF}{\tilde{\sty{ F}}}
\newcommand{\fG}{\sty{ G}}
\newcommand{\fH}{\sty{ H}}
\newcommand{\fI}{\sty{ I}}
\newcommand{\fJ}{\sty{ J}}
\newcommand{\fK}{\sty{ K}}
\newcommand{\fL}{\sty{ L}}
\newcommand{\fM}{\sty{ M}}
\newcommand{\fN}{\sty{ N}}
\newcommand{\fO}{\sty{ 0}}
\newcommand{\fP}{\sty{ P}}
\newcommand{\fQ}{\sty{ Q}}
\newcommand{\fR}{\sty{ R}}
\newcommand{\fS}{\sty{ S}}
\newcommand{\fT}{\sty{ T}}
\newcommand{\fU}{\sty{ U}}
\newcommand{\fV}{\sty{ V}}
\newcommand{\fW}{\sty{ W}}
\newcommand{\fX}{\sty{ X}}
\newcommand{\fY}{\sty{ Y}}
\newcommand{\fZ}{\sty{ Z}}
%
\newcommand{\bfA}{\bar{\sty{ A}}}
\newcommand{\bfB}{\bar{\sty{ B}}}
\newcommand{\bfC}{\bar{\sty{ C}}}
\newcommand{\bfD}{\bar{\sty{ D}}}
\newcommand{\bfE}{\bar{\sty{ E}}}
\newcommand{\bfF}{\bar{\sty{ F}}}
\newcommand{\bfG}{\bar{\sty{ G}}}
\newcommand{\bfH}{\bar{\sty{ H}}}
\newcommand{\bfI}{\bar{\sty{ I}}}
\newcommand{\bfJ}{\bar{\sty{ J}}}
\newcommand{\bfK}{\bar{\sty{ K}}}
\newcommand{\bfL}{\bar{\sty{ L}}}
\newcommand{\bfM}{\bar{\sty{ M}}}
\newcommand{\bfN}{\bar{\sty{ N}}}
\newcommand{\bfO}{\bar{\sty{ 0}}}
\newcommand{\bfP}{\bar{\sty{ P}}}
\newcommand{\bfQ}{\bar{\sty{ Q}}}
\newcommand{\bfR}{\bar{\sty{ R}}}
\newcommand{\bfS}{\bar{\sty{ S}}}
\newcommand{\bfT}{\bar{\sty{ T}}}
\newcommand{\bfU}{\bar{\sty{ U}}}
\newcommand{\bfV}{\bar{\sty{ V}}}
\newcommand{\bfW}{\bar{\sty{ W}}}
\newcommand{\bfX}{\bar{\sty{ X}}}
\newcommand{\bfY}{\bar{\sty{ Y}}}
\newcommand{\bfZ}{\bar{\sty{ Z}}}
\newcommand{\bfeps}{\bar{\feps}}
\newcommand{\bfsigma}{\bar{\fsigma}}
\newcommand{\bftau}{\bar{\ftau}}
%
\newcommand{\ffA}{\styy{ A}}
\newcommand{\ffB}{\styy{ B}}
\newcommand{\ffC}{\styy{ C}}
\newcommand{\ffD}{\styy{ D}}
\newcommand{\ffE}{\styy{ E}}
\newcommand{\ffF}{\styy{ F}}
\newcommand{\ffG}{\styy{ G}}
\newcommand{\ffH}{\styy{ H}}
\newcommand{\ffI}{\styy{ I}}
\newcommand{\ffJ}{\styy{ J}}
\newcommand{\ffK}{\styy{ K}}
\newcommand{\ffL}{\styy{ L}}
\newcommand{\ffM}{\styy{ M}}
\newcommand{\ffN}{\styy{ N}}
\newcommand{\ffO}{\styy{ O}}
\newcommand{\ffP}{\styy{ P}}
\newcommand{\ffQ}{\styy{ Q}}
\newcommand{\ffR}{\styy{ R}}
\newcommand{\ffS}{\styy{ S}}
\newcommand{\ffT}{\styy{ T}}
\newcommand{\ffU}{\styy{ U}}
\newcommand{\ffV}{\styy{ V}}
\newcommand{\ffW}{\styy{ W}}
\newcommand{\ffX}{\styy{ X}}
\newcommand{\ffY}{\styy{ Y}}
\newcommand{\ffZ}{\styy{ Z}}
%
\newcommand{\falpha}{\mbox{\boldmath $\alpha$}}
\newcommand{\fchi}{\mbox{\boldmath $\chi$}}
\newcommand{\fdelta}{\mbox{\boldmath $\delta$}}
\newcommand{\ftau}{\mbox{\boldmath $\tau$}}
\newcommand{\fsigma}{\mbox{\boldmath $\sigma$}}
\newcommand{\fDelta}{\mbox{\boldmath $\Delta$}}
\newcommand{\fomega}{\mbox{\boldmath $\omega$}}
\newcommand{\fOmega}{\mbox{\boldmath $\Omega$}}
\newcommand{\fLambda}{\mbox{\boldmath $\Lambda$}}
\newcommand{\fxi}{\mbox{\boldmath $\xi $}}
\newcommand{\fXi}{\mbox{\boldmath $\Xi $}}
\newcommand{\feps}{\mbox{\boldmath $\varepsilon $}}
\newcommand{\fgamma}{\mbox{\boldmath $\gamma $}}
\newcommand{\fGamma}{\mbox{\boldmath $\Gamma $}}
\newcommand{\fSigma}{\mbox{\boldmath $\Sigma $}}
\newcommand{\fPi}{\mbox{\boldmath $\Pi $}}
\newcommand{\fPhi}{\mbox{\boldmath $\Phi $}}
\newcommand{\fphi}{\mbox{\boldmath $\phi $}}
\newcommand{\fvarphi}{\mbox{\boldmath $\varphi $}}
\newcommand{\fPsi}{\mbox{\boldmath $\Psi $}}
\newcommand{\fJota}{{\bf I}}
\newcommand{\fepsi}{\mbox{\boldmath $\epsilon $}}
%
\newcommand{\cA}{{\cal A}}
\newcommand{\cB}{{\cal B}}
\newcommand{\cC}{{\cal C}}
\newcommand{\cD}{{\cal D}}
\newcommand{\cE}{{\cal E}}
\newcommand{\cF}{{\cal F}}
\newcommand{\cG}{{\cal G}}
\newcommand{\cH}{{\cal H}}
\newcommand{\cI}{{\cal I}}
\newcommand{\cJ}{{\cal J}}
\newcommand{\cK}{{\cal K}}
\newcommand{\cL}{{\cal L}}
\newcommand{\cM}{{\cal M}}
\newcommand{\cN}{{\cal N}}
\newcommand{\cO}{{\cal O}}
\newcommand{\cP}{{\cal P}}
\newcommand{\cQ}{{\cal Q}}
\newcommand{\cR}{{\cal R}}
\newcommand{\cS}{{\cal S}}
\newcommand{\cT}{{\cal T}}
\newcommand{\cU}{{\cal U}}
\newcommand{\cV}{{\cal V}}
\newcommand{\cW}{{\cal W}}
\newcommand{\cX}{{\cal X}}
\newcommand{\cY}{{\cal Y}}
\newcommand{\cZ}{{\cal Z}}
%%%%%%%%%%%%%%%%%%%%added by Julian Bauer
%vector
\newcommand{\vb}{\mbox{$\underline{\smash{b}}$}}
\newcommand{\vf}{\mbox{$\underline{\smash{f}}$}}
\newcommand{\vp}{\mbox{$\underline{\smash{p}}$}}
\newcommand{\vx}{\mbox{$\underline{\smash{x}}$}}
%\newcommand{\vn}{\mbox{$\underline{\smash{n}}$}}
%\newcommand{\vA}{\mbox{$\underline{\smash{A}}$}}
%\newcommand{\vB}{\mbox{$\underline{\smash{B}}$}}
%\newcommand{\vC}{\mbox{$\underline{\smash{C}}$}}
%\newcommand{\vD}{\mbox{$\underline{\smash{D}}$}}
%\newcommand{\vE}{\mbox{$\underline{\smash{E}}$}}
%\newcommand{\vS}{\mbox{$\underline{\smash{S}}$}}
\newcommand{\vn}{\fn}
\newcommand{\vA}{\fa}
\newcommand{\vB}{\fb}
\newcommand{\vC}{\fc}
\newcommand{\vD}{\fd}
\newcommand{\vE}{\fe}
\newcommand{\vS}{\fs}
%matrix
\newcommand{\mA}{\mbox{$\underline{\underline{\smash{A}}}$}}

%Mengen
\newcommand{\Nnatural}{\mbox{$\mathbb{N}$}}
\newcommand{\Rreal}{\mbox{$\mathbb{R}$}}

%specific
\newcommand{\aAOI}{\mbox{$a_{\text{aoi}}$}}
\newcommand{\Zver}{\mbox{$Z_{\text{ver}}$}}
\newcommand{\Ztotal}{\mbox{$Z_{\text{total}}$}}
\newcommand{\Zfmax}{\mbox{$Z_{\text{fmax}}$}}
\newcommand{\Zfpmax}{\mbox{$Z_{\text{fkmax}}$}}
\newcommand{\Zfav}{\mbox{$Z_{\text{fav}}$}}
\newcommand{\Zbiax}{\mbox{$Z_{\text{biax}}$}}
\newcommand{\Zhom}{\mbox{$Z_{\text{hom}}$}}
\newcommand{\IVOL}{\mbox{$V_{\text{ip}}$}}


\newcommand{\gfmax}{\mbox{$\gamma_{\text{fmax}}$}}
\newcommand{\gfpmax}{\mbox{$\gamma_{\text{fkmax}}$}}
\newcommand{\gfav}{\mbox{$\gamma_{\text{fav}}$}}
\newcommand{\gbiax}{\mbox{$\gamma_{\text{biax}}$}}
\newcommand{\ghom}{\mbox{$\gamma_{\text{hom}}$}}

\newcommand{\failp}{\mbox{$k$}}
\newcommand{\fPS}{\mbox{$\varepsilon_1\left(\fx\right)$}}
\newcommand{\sPS}{\mbox{$\varepsilon_2\left(\fx\right)$}}

\newcommand{\eMacro}{\mbox{$\varepsilon$}}%_{\text{macro}}$}}
\newcommand{\eMacroZero}{\mbox{$\varepsilon_0$}}%_{\text{macro}}$}}
\newcommand{\FMacro}{\mbox{$F_{13}$}}%_{\text{macro}}$}}
\newcommand{\ePMacro}{\mbox{$\dot{\varepsilon}$}}%_{\text{macro}}$}}
\newcommand{\deltaEMacro}{\mbox{$\triangle\varepsilon$}}%_{\text{macro}}$}}
\newcommand{\deltaT}{\mbox{$\triangle t$}}


\newcommand{\FmaxBiaxHom}{\mbox{FmaxBiaxHom$15$}}
\newcommand{\FmaxBiaxHomOut}{\mbox{FmaxBiaxHom$10$Out}}
\newcommand{\FavBiaxHom}{\mbox{FavBiaxHom$15$}}
\newcommand{\FmaxFavBiaxHom}{\mbox{FmaxFavBiaxHom$15$}}
\newcommand{\FpmaxBiaxHomTen}{\mbox{FkmaxBiaxHom$10$}}
\newcommand{\FpmaxBiaxHomFiveTen}{\mbox{FkmaxBiaxHom$15$}}

\newcommand{\led}{\mbox{$l_{13}\left(t\right)$}}
\newcommand{\ledn}{\mbox{$l_{13}\left(t=0\right)$}}
\newcommand{\lzv}{\mbox{$l_{24}\left(t\right)$}}
\newcommand{\lzvn}{\mbox{$l_{24}\left(t=0\right)$}}


%%%%%%%%%%%%%%%%%%%%%%%%%%%%%%%%%%%%%%%%%%%%%%%%%%%%%%%%%%%%%%%%%%%%%%%
% Tikz functions

\newcommand{\drawBasis}[5]{%
    %1 scaleFont
    %2 scaleLineWidth
    %3 label x-axis
    %4 label y-axis
    %5 label z-axis
    \tikzmath{  \sLW=#2;}
    \draw[line width=1pt*\sLW,->] (0,0,0) -- (1,0,0) node[scale=#1, anchor=north east]{$#3$};
    \draw[line width=1pt*\sLW,->] (0,0,0) -- (0,1,0) node[scale=#1, anchor=south]{$#4$};
    \draw[line width=1pt*\sLW,->] (0,0,0) -- (0,0,1) node[scale=#1, anchor=south]{$#5$};
}

\newcommand{\clock}[6]{%
    \begin{scope}[xshift=#1, yshift=#2, scale=#3, line cap=round]
        \tikzmath{
            \sLW = #4;  % Scale line widths
        } 
        \draw [line width=1.6pt*\sLW] (0,0) circle (1.15cm);
        \foreach \angle in {0, 30, ..., 330} 
            \draw[line width=1pt*\sLW] (\angle:0.89cm) -- (\angle:1cm);
        \foreach \angle in {0,90,180,270}
            \draw[line width=1.4pt*\sLW] (\angle:0.82cm) -- (\angle:1cm);
        \draw[line width=1.6pt*\sLW] (0,0) -- (90-30*#5:0.4cm); % hour-hand
        \draw[line width=1.3pt*\sLW] (0,0) -- (90-6*#6:0.65cm); % minute-hand
    \end{scope}
}

\newcommand{\guy}[4]{%
    \begin{scope}[xshift=#1,yshift=#2,scale=#3]
        \tikzmath{
            \sLW = #4;  % Scale line widths
        } 
        \draw[line width=2pt*\sLW] (0,0) circle (1cm);               % head
        \draw[fill=black] (35:0.3cm) circle (0.08cm*\sLW);           % right eye
        \draw[fill=black] (145:0.3cm) circle (0.08cm*\sLW);          % left eye
        \draw[line width=2pt*\sLW] ([shift={+(0cm,0cm)}]245:0.5cm) arc (245:295:0.5cm);
        
        \draw[line width=2pt*\sLW] (0,-1cm) -- ++(-90:2.25cm);       % body 
        \draw[line width=2pt*\sLW] (0,-2cm) -- ++(30:1.5cm);         % right arms
        \draw[line width=2pt*\sLW] (0,-2cm) -- ++(150:1.5cm);        % left arms
        \draw[line width=2pt*\sLW] (0,-3.25cm) -- ++(-60:1.5cm);     % rights leg
        \draw[line width=2pt*\sLW] (0,-3.25cm) -- ++(240:1.5cm);     % left leg
    \end{scope}
}

\newcommand{\rail}[5]{%
    \begin{scope}[xshift=#1, yshift=#2, scale=#3]
        \tikzmath{
            \sLW = #4;
        }
        \draw[line width=2pt*\sLW] (-0.1cm,0) -- (#5,0);
        \draw[line width=1pt*\sLW, dashed] (0,-0.1cm) node[below]{Ziel} -- (0,2cm) ;
    \end{scope}
}

\newcommand{\locomotive}[4]{%
    \begin{scope}[xshift=#1, yshift=#2, scale=#3]
        \tikzmath{
            \sLW = #4;
        }
        \draw[line width=2pt*\sLW] (0,0.3cm) 
            -- ++(6cm,  0) 
            -- ++(0,    1.4cm) 
            -- ++(-5cm, 0) 
            --   (0,    0.3cm);
        \draw[line width=2pt*\sLW] (0.5cm,0.9cm) -- ++(1.5cm,0) -- (2.0cm,1.7cm); 
        \foreach \wheel in {1,2,3.5,4.5,5.5}{
            \draw [line width=2pt*\sLW]([shift={(\wheel,0.3cm)}]180:0.3cm) arc (180:360:0.3cm);
        }
    \end{scope}
}

\newcommand{\railCar}[4]{%
    \begin{scope}[xshift=#1, yshift=#2, scale=#3]
        \tikzmath{
            \sLW = #4;
        } 
        \draw[line width=2pt*\sLW] 
                  (0,   0.3cm) 
            --  ++(5cm, 0)
            --  ++(0,   1.4cm)
            --  ++(-5cm,0)
            --    (0,   0.3cm);
        \foreach \wheel in {0.8,1.8,3.2,4.2}{
            \draw [line width=2pt*\sLW]
                ([shift={(\wheel,0.3cm)}]180:0.3cm) arc (180:360:0.3cm);
        }
    \end{scope}
}

\newcommand{\train}[4]{%
    \begin{scope}[shift={++(#1,#2)}, scale=#3]
        \rail{0}{0}{#3}{#4}{11.5cm};
        \locomotive{0}{0}{#3}{#4}; 
        \railCar{6.5cm*#3}{0}{#3}{#4};
    \end{scope}
}

\newcommand{\spaceTime}[9]{%
    %1 xshift
    %2 yshift
    %3 scale
    %4 scaleLineWidth
    %5 scaleFont
    %6 label x
    %7 label y
    %8 label z
    %9 label t
    \begin{scope}[xshift=#1, yshift=#2, scale=#3]
        \tdplotsetmaincoords{55}{30}
        \begin{scope}[scale=1, tdplot_main_coords]
            \drawBasis{#5}{#4}{#6}{#7}{#8}
            \begin{scope}[scale=2]
                \clock{0.43cm}{0.1cm}{0.15}{0.8*#4}{9}{5};
                \begin{scope}[xshift=0.7cm, yshift=0]    
                    \node[scale=#5](0,0){$#9$};
                \end{scope}
            \end{scope}
        \end{scope}
        \tdplotsetmaincoords{0}{0}
    \end{scope}
}

\newcommand{\boxedBasis}[6]{%
    %1 label x-axis
    %2 label y-axis
    %3 label z-axis
    %4 label box 
    %5 scaleLineWidth
    %6 scaleFont
    \draw[line width=1pt*#5,->] (0,0)--(1,0)
        node[transform shape, scale=#6, anchor=north]      {#1};

    \draw[line width=1pt*#5,->] (0,0)--(0.5,0.5)
        node[transform shape, scale=#6, anchor=west]       {#2};

    \draw[line width=1pt*#5,->] (0,0)--(0,1)
        node[transform shape, scale=#6, anchor=east]       {#3};

    \tikzmath{\offset=0.6;}

    \draw[line width=1pt*#5, draw=black] 
        (-\offset, -\offset) rectangle 
        ( 1+0.5*\offset, 1+0.5*\offset) 
        node[   midway, 
                label={[label distance=0.8cm, transform shape]0:{#4}}]{};
}

%%%%%%%%%%%%%%%%%%%%%%%%%%%%%%%%%%%%%%%%%%%%%%%%%%%%%%%%%%%%%%%%%%%%%%%
% Title
\begin{document}
\title{
        Albert Einsteins \glqq{}Spezielle Relativitätstheorie\grqq{} 
        einfach präsentiert
}
\author[Julian Bauer]{
        \begin{tabular}{c} 
                Julian Bauer 
        \end{tabular}
}
\institute{Karlsruhe Institute of Technology (KIT)}
\date[]{13.06.2018}

\begin{frame}
    \titlepage
\end{frame}

%%%%%%%%%%%%%%%%%%%%%%%%%%%%%%%%%%%%%%%%%%%%%%%%%%%%%%%%%%%%%%%%%%%%%%%
% Table of contents
\begin{frame}
    \tableofcontents
\end{frame}

%%%%%%%%%%%%%%%%%%%%%%%%%%%%%%%%%%%%%%%%%%%%%%%%%%%%%%%%%%%%%%%%%%%%%%%
\section{Einleitung}
\begin{frame}
    \tableofcontents[hideallsubsections,currentsection,currentsubsection]
\end{frame}

\begin{frame}
	\frametitle{Motivation}
    \begin{itemize}
    	\item   Zur damaligen Zeit versuchten theoretische Physiker die Welt durch 
                Einführung eines \textbf{absolut ruhenden Systems} zu erklären.
                Dieses absolut ruhende System wurde \textbf{Lichtäther} genannt und
                konnte nie nachgewiesen werden.
                \vspace{10pt}
        \item   \textbf{Unbefriedigende Formulierungen} der damaligen Grundgleichungen 
                der Elektrodynamik.
        \begin{itemize}
			\item   Beispiel:\\
                        Eine Bewegung eines elektrischen Leiters relativ zu einem
                        Magneten führt zu elektrischen Strömen im Leiter.\\
                        Damalige Erklärungen unterscheiden \textbf{zwei Fälle}:
                        \begin{enumerate}
                        	\item Der Leiter steht fest.
                            \item Der Magnet steht fest
                        \end{enumerate}
		\end{itemize}
	\end{itemize}
\end{frame}

\begin{frame}
	\href{https://pep20.org/}{PEP20.org} folgend 
    \begin{figure}[h]
            \centering
            \includegraphics[width = 0.7\textwidth]{pep20}\\
            \dots
    \end{figure}
    hat A.E. eine einfache (komplexe) 
    Lösung für die Elektrodynamik gesucht und in
    \glqq Zur Elektrodynamik bewegter Körper\grqq{} formuliert.
\end{frame}

\begin{frame}
    \frametitle{Struktur von \glqq{} Zur Elektrodynamik bewegter Körper \grqq{}}
    \begin{enumerate}[I]
        \item Kinematischer Teil (15 Seiten)
        \begin{itemize}
			\item[§1]   Definition von \textbf{Gleichzeitigkeit}
			\item[§2]   Über die \textbf{Relativität von Längen und Zeiten}
			\item[§3]   Theorie der \textbf{Koordinaten- und Zeittransformation} von dem ruhenden
                        auf ein relativ zu diesem in gleichförmiger Translationsbewegung 
                        befindliches System
			\item[§4]   \textbf{Physikalischer Bedeutung} der erhaltenen Gleichungen, 
                        bewegte starre Körper und bewegte Uhren betreffend
			\item[§5]   Additionstheoreme der Geschwindigkeiten
		\end{itemize}
        \vspace{10pt}
		\item Elektrodynamischer Teil (15 Seiten)
        \begin{itemize}
            \item[]\dots
        \end{itemize}
	\end{enumerate}
\end{frame}
\section{§1 Gleichzeitigkeit}

\begin{frame}
    \tableofcontents[hideallsubsections,currentsection,currentsubsection]
\end{frame}

\begin{frame}{Raum}
    \begin{itemize}
        \item Das \textbf{\glqq{}ruhende System\grqq{}:}
        \begin{itemize}
            \item Die Newtonschen mechanischen Gleichungen gelten
            \begin{itemize}
                \item   Keine Kraft $\rightarrow$ keine Beschleunigung
                \item   Kraft = Masse * Beschleunigung
                \item   Actio = Reactio
            \end{itemize}
            \item       Der \textbf{Ort} eines ruhenden Punktes im ruhenden System
                        kann über Koordinaten 
                        bestimmt werden. Diese Koordinaten stellen Vielfache starrer 
                        Maßstäbe dar.
        \end{itemize}
    \end{itemize}
\end{frame}

\begin{frame}
\begin{columns}
\begin{column}{0.5\textwidth}
    \begin{figure}[h]
        \centering
        % Set the plot display orientation
        \tdplotsetmaincoords{60}{20}
        % Use display orientation
        \begin{tikzpicture}[scale=5,tdplot_main_coords]
            \tikzmath{
                \labelDistance=1;
                \rvec=1.2;
                \thetavec=50;
                \phivec=60;
            }
            
            % Define P by polar coordinates
            \tdplotsetcoord{P}{\rvec}{\thetavec}{\phivec}
            
            % Draw the main coordinate system axes
            \drawBasis{1}{1}{x}{y}{z};
            
            % Draw a vector from origin to point (P) 
            \draw (P) node[ cross=8pt, 
                            line width=0.7mm,
                            label={[label distance=10pt]above:{Ruhender Punkt}}
                            ]{};
            
            % Draw projections
            \draw[dashed, color=blue, stealth-stealth]
                    (P) -- node[
                        label={[label distance=1pt*\labelDistance]right:{z}}]{}
                    (Pxy);
            
            \draw[dashed, color=blue,stealth-stealth]
                    (Pxy) -- node[
                        label={[label distance=1pt*\labelDistance]right:{y}}]{}
                    (Px);
            
            \draw[dashed, color=blue,stealth-stealth]
                    (Pxy) -- node[
                        label={[label distance=1pt*\labelDistance]above:{x}}]{}
                    (Py);
            
            % Draw ein Massstab
            \draw[thick, color=green, stealth-stealth]
                    (0.2,-0.1,0) -- node[
                        label={[label distance=1pt*\labelDistance]below:{\text{Maßstab}}}]{}
                    (0.3,-0.1,0);    
        \end{tikzpicture}
    \end{figure}
\end{column}
\begin{column}{0.5\textwidth}  %%<--- here
    \centering
    \mbox{$x =$ x-fache Länge des Maßstabs}\\
    \vspace{5pt}
    \dots
\end{column}
\end{columns}
\end{frame}

\begin{frame}
    \begin{itemize}
        \item Die \textbf{Bewegung} eines Körpers ist\\
            \begin{center}
                der Ort des Körpers als Funktion der \textbf{Zeit}
            \end{center}
    \end{itemize}
    \begin{figure}[h]
        \centering
        % Set the plot display orientation
        \tdplotsetmaincoords{55}{30}

        % Use display orientation
        \begin{tikzpicture}[scale=4,tdplot_main_coords]
            \tikzmath{
                \labelDistance=1;
                \rvec=1.2;
                \thetavec=50;
                \phivec=60;
            }
            % Draw Basis
            \drawBasis{1}{1}{x}{y}{z}

            % Define coordinate previous staring with origin
            \coordinate (previous) at (0,0,0);
            \foreach \step in {0.1, 0.2, ..., 1.0}{
                \tdplotsetcoord{P}{\rvec*\step}{\thetavec}{\phivec*\step*\step}

                \draw (P) node[ cross=4pt, 
                                line width=0.4mm,
                                ]{};
        
                % Draw projections
                \draw[dashed, color=blue, stealth-stealth]
                        (P) -- (Pxy);
                \draw[dashed, color=blue,stealth-stealth]
                        (Pxy) -- (Px);
                \draw[dashed, color=blue,stealth-stealth]
                        (Pxy) -- (Py);

                % Connect points
                \draw[thick, red] (previous) -- (P); 

                % Update previous
                \coordinate (previous) at (P);
            }
        \end{tikzpicture}
    \end{figure}

\end{frame}

\begin{frame}
    \begin{itemize}
        \item Ein \textbf{Ereignis} ist gekennzeichnet durch\\
            \begin{center}
                einen \textbf{Ort} und eine \textbf{Zeit}
            \end{center}
        \visible<2->{
        \item Beispiele:
        \begin{itemize}
            \item Wir sind \textbf{hier} und \textbf{jetzt}.
            \vspace{5pt}
        }
        \visible<3->{
            \item Der Körper ist \textbf{jetzt dort}.
        }
        \end{itemize}
    \end{itemize}
\end{frame}

\begin{frame}
    \begin{itemize}
        \item Die \textbf{Bewegung} eines Körpers kann als\\
            \begin{center}
                \textbf{Abfolge von Ereignissen} interpretiert werden.
            \end{center}
    \end{itemize}
\end{frame}

\begin{frame}
    \begin{itemize}
        \item   Wir definieren Zeit über \textbf{Gleichzeitigkeit}\\
            \vspace{5pt}
            \begin{figure}[h]
                    \centering
                    \includegraphics[width = 0.9\textwidth]{gleichzeitigkeit2}\\
                    \caption{\cite{Einstein1905}}
            \end{figure}
            \vspace{5pt}
    \end{itemize}
\end{frame}

\begin{frame}{Gleichzeitigkeit}[fragile]
    %fragile-option is needed if macros are to be executed in beamer frame
    % see  https://tex.stackexchange.com/a/325472
    \begin{figure}[h]
        \centering
        \begin{tikzpicture}
            \clock{0}{0}{0.8}{0.8}{7}{0};    
            \guy{4cm}{0}{0.5}{0.5};
            \train{0}{-4cm}{0.7}{0.5};
            \draw[thick,->] (3,   0) -- (1.5,     0);
            \draw[thick,->] (3,  -1) -- (2,      -2);
        \end{tikzpicture}
    \end{figure}

\end{frame}

\begin{frame}
    \begin{itemize}
        \item   Wir definieren Zeit über \textbf{Gleichzeitigkeit}\\
            \vspace{5pt}
            \begin{figure}[h]
                    \centering
                    \includegraphics[width = 0.9\textwidth]{gleichzeitigkeit2}\\
                    \caption{\cite{Einstein1905}}
            \end{figure}
            \vspace{5pt}
        \item   Der Beobachter ist direkt am Ort des Ereignisses.
    \end{itemize}
\end{frame}

\begin{frame}{Raumzeit}
    \begin{figure}[h]
        \centering
        \begin{equation*}
            \text{Ereignis} = 
                \begin{pmatrix} 
                    x\\
                    y\\
                    z\\
                    t\\
                \end{pmatrix}_{ \left\{
                    \text{Koordinatensystem}, \textbf{ lokale ruhende Uhr}
                                \right\}}
        \end{equation*}
        \begin{tikzpicture}
            \spaceTime{-3.5cm}{0cm}{1.5}{1.5}{1.2}{x}{y}{z}{t};
            \train{0}{0}{0.7}{0.8};
        \end{tikzpicture}
    \end{figure}        
\end{frame}

\begin{frame}{Aufgabe: Synchrone ruhende Uhren}
    \begin{figure}[h]
        \centering
        \begin{tikzpicture}
            \spaceTime{-3cm}{0cm}{1.5}{1.5}{1.2}{x_A}{y_A}{z_A}{t_A};
            \draw[line width=2pt, stealth-stealth] 
                (-0.2cm,0) -- (2.2cm,0)
                node[midway, above]{?};               
            \spaceTime{3cm}{0cm}{1.5}{1.5}{1.2}{x_B}{y_B}{z_B}{t_B};
            \visible<2->{
            \guy{-2.5cm}{-2cm}{0.2}{0.5};
            \node at (-2cm, -2cm) {A};
            \guy{3.5cm}{-2cm}{0.2}{0.5};
            \node at (4cm, -2cm) {B};
            }
        \end{tikzpicture}
    \end{figure}        
\end{frame}

\begin{frame}{Aufgabe: Synchrone ruhende Uhren}
    \begin{figure}[h]
            \centering
            \includegraphics[width = \textwidth]{problemSyncron}
            \caption{\cite{Einstein1905}}
    \end{figure}
\end{frame}

\begin{frame}{Lösung: Definition}
    \begin{itemize}
        \item Die letztere Zeit = Die Zeit = Eine gemeinsame Zeit. 
    \end{itemize}
    \begin{figure}[h]
            \centering
            \includegraphics[width = \textwidth]{loesungSyncron}
            \caption{\cite{Einstein1905}}
    \end{figure}
\end{frame}

\begin{frame}{Synchrone ruhende Uhren}
    \begin{figure}[h]
        \centering
        \begin{tikzpicture}
            \spaceTime{-3cm}{0cm}{1.5}{1.5}{1.2}{x_A}{y_A}{z_A}{t_A};
            \spaceTime{3cm}{0cm}{1.5}{1.5}{1.2}{x_B}{y_B}{z_B}{t_B};
            \tikzmath{
                \xA = -2.5;
                \xB = 4;
                \yStart = -1.5;
                \dY = -0.5;
                \lD = 0.5;
            }
            \draw (\xA,-1.4cm) -- ++(0,-2.0);
            \draw (\xB,-1.4cm) -- ++(0,-2.0cm);
            % Draw distance 
            \draw [line width = 1pt, stealth-stealth] 
                (\xA,\yStart+\dY) 
                -- (\xB,\yStart+\dY) 
                node[midway, above]{$\overline{AB}$};
            % Draw light forward
            \draw [line width = 2pt, ->,red] 
                (\xA,\yStart+\dY*2) 
                node [label={[label distance=\lD,black]left:{$t_A$}}]{}
                -- (\xB,\yStart+\dY*2) 
                node [label={[label distance=\lD,black]right:{$t_B$}}]{};
            % Draw light backward
            \draw [line width = 2pt, ->,red] 
                (\xB,\yStart+\dY*3) 
                node [label={[label distance=\lD,black]right:{}}]{}
                -- (\xA,\yStart+\dY*3) 
                node [label={[label distance=\lD,black]left:{$t^{\prime}_A$}}]{};
        \end{tikzpicture}
    \end{figure}
    \only<2>{
        Synchronisationsbedingung:       
        \begin{equation*}
            t_B - t_A = t^{\prime}_A - t_B
        \end{equation*}
    }
    \only<3>{
        Bestimmung der Lichtgeschwindigkeit im leeren Raum:
        \begin{equation*}
            V=\frac{2\overline{AB}}{t^{\prime}_A - t_A}
        \end{equation*}
    }
\end{frame}

\begin{frame}{Die Zeit}
    \begin{figure}[h]
            \centering
            \includegraphics[width = \textwidth]{dieZeit}
            \caption{\cite{Einstein1905}}
    \end{figure}
\end{frame}

\section{§2 Relativität Zeiten \& Längen}

\begin{frame}
    \tableofcontents[hideallsubsections,currentsection,currentsubsection]
\end{frame}

\begin{frame}{Wo sind wir jetzt?}
    \begin{itemize}
        \item Bisher:
            \begin{itemize}
                \item Definition von Gleichzeitigkeit
            \end{itemize}
        \vspace{5pt}
        \item Jetzt:
            \begin{itemize}
                \item Zwei  Prinzipien und deren Folgen für:
                    \begin{center}
                        Die Relativität von Längen und Zeiten
                    \end{center}
            \end{itemize}
        \vspace{5pt}
        \item Danach:
            \begin{itemize}
                \item Formeln
            \end{itemize}
    \end{itemize}
\end{frame}

\begin{frame}{Relativitätsprinzip}
    \begin{figure}[h]
            \centering
            \includegraphics[width = \textwidth]{relativitaet}
            \caption{\cite{Einstein1905}}
    \end{figure}
\end{frame}

\begin{frame}
    \begin{tikzpicture}
        \begin{scope}[xshift=0cm, yshift=0cm, rotate=30]
            \boxedBasis{$x_1$}
                        {$y_1$}
                        {$z_1$}
                        {$\rightarrow \boldsymbol{v_1}=\text{konstant}$}
                        {1}
                        {1};
        \end{scope}
        \begin{scope}[xshift=5cm, yshift=0cm, rotate=10]
            \boxedBasis{$x_2$}
                        {$y_2$}
                        {$z_2$}
                        {$\rightarrow \boldsymbol{v_2}=\text{konstant}$}
                        {1}
                        {1};
        \end{scope}
        \begin{scope}[xshift=3cm, yshift=-3cm, rotate=-20]
            \boxedBasis{$x_3$}{$y_3$}{$z_3$}{}{1}{1};
        \end{scope}
    \end{tikzpicture}
\end{frame}

\begin{frame}{Konstanz der Lichtgeschwindigkeit}
    \begin{figure}[h]
            \centering
            \includegraphics[width = \textwidth]{konstanzLichtgeschwindigkeit}
            \caption{\cite{Einstein1905}}
    \end{figure}
\end{frame}

\begin{frame}{Konstanz der Lichtgeschwindigkeit}
    \begin{figure}[h]
        \centering
        \includegraphics[width = \textwidth]{konstGeschwRuhend}
        \caption{Ausbreitungsgeschwindigkeit des Lichtes einer \textbf{ruhenden} Lichtquelle}
    \end{figure}
\end{frame}

\begin{frame}{Konstanz der Lichtgeschwindigkeit}
    \begin{figure}[h]
        \centering
        \includegraphics[width = \textwidth]{konstGeschwBewegt}
        \caption{Ausbreitungsgeschwindigkeit des Lichtes einer \textbf{bewegten} Lichtquelle}
    \end{figure}
\end{frame}

\begin{frame}{Bewegte Uhr synchron im ruhenden System}
    \begin{figure}[h]
        \centering
        \includegraphics[width = 0.8\textwidth]{synchronRuhend}
        \caption{\textbf{SR} $=$ synchron im ruhenden System}
    \end{figure}
\begin{itemize}
    \item Die Zeit einer bewegten Uhr (SR)  entspricht der Zeit einer ruhenden synchronen Uhr am aktuellen Ort der bewegten Uhr.
\end{itemize}
\end{frame}

\begin{frame}{Relativität von Zeiten}
    \begin{figure}[h]
        \centering
        \includegraphics[width = 0.8\textwidth]{relZeitOhne}
    \end{figure}
\end{frame}

\begin{frame}{Relativität von Zeiten}
    \begin{figure}[h]
        \centering
        \includegraphics[width = 0.8\textwidth]{relZeitMit}
    \end{figure}
\end{frame}

\begin{frame}
    \begin{figure}[h]
        \centering
        \includegraphics[width = 0.7\textwidth]{relZeitMit}
    \end{figure}
    \begin{figure}[h]
        \centering
        \begin{tikzpicture}
            \tikzmath{
                \xA = -2.5;
                \xB = 4;
                \yStart = -1.5;
                \dY = -0.5;
                \lD = 0.5;
            }
            \draw (\xA,\yStart+\dY) -- ++(0,-1.5);
            \draw (\xB,\yStart+\dY) -- ++(0,-1.5cm);
            % Draw light forward
            \draw [line width = 2pt, ->,red] 
                (\xA,\yStart+\dY*2) 
                node [label={[label distance=\lD,black]left:{$t_A$}}]{}
                -- (\xB,\yStart+\dY*2) 
                node [label={[label distance=\lD,black]right:{$t_B$}}]{};
            % Draw light backward
            \draw [line width = 2pt, ->,red] 
                (\xB,\yStart+\dY*3) 
                node [label={[label distance=\lD,black]right:{}}]{}
                -- (\xA,\yStart+\dY*3) 
                node [label={[label distance=\lD,black]left:{$t^{\prime}_A$}}]{};
        \end{tikzpicture}
    \end{figure}
    \only<2->{
        Synchronisationsbedingung:       
        \begin{align*}
            t_B - t_A               &= t^{\prime}_A - t_B\\
            \visible<3->{
                \frac{r_{AB}}{V-v}  &= \frac{r_{AB}}{V+v} 
            }
        \end{align*}
    }
\end{frame}

\begin{frame}
    \begin{figure}[h]
        \centering
        \includegraphics[width = 0.7\textwidth]{relZeitMit}
    \end{figure}
        Synchronisationsbedingung:       
        \begin{align*}
            t_B - t_A           &= t^{\prime}_A - t_B\\
            \frac{r_{AB}}{V-v}  &= \frac{r_{AB}}{V+v} 
        \end{align*}
Interpretation:
\begin{itemize}
    \item Bewegte Beobachter sehen zwei bewegte Uhren (\textbf{SR}), nicht synchron.
    \visible<2->{
        \item Gleichzeitigkeit ist relativ 
        \visible<3->{
            $=$ \textbf{Zeit ist relativ}
        }
    }
\end{itemize}
\end{frame}

\begin{frame}{Relativität von Längen}
    \begin{columns}
        \begin{column}{0.5\textwidth}
            \begin{figure}[h]
                \centering
                \includegraphics[height = 0.6\textwidth]{relLaengeRuhend}
            \end{figure}
        \end{column}
        \begin{column}{0.5\textwidth}
            \begin{figure}[h]
                \centering
                \includegraphics[height = 0.6\textwidth]{relLaengeBewegt}
            \end{figure}
        \end{column}
    \end{columns}
    \visible<2->{
        \begin{itemize}
            \item A.E.: Relativitätsprinzip:
                \begin{equation*}
                    l_{\text{ruhend}} = l_{\text{bewegt}}
                \end{equation*}
            \vspace{-20pt}
            \visible<3->{
                \item \textbf{Aber}: Die Länge des bewegten Stabes im ruhenden System $\neq$ $l$ 
            }
        \end{itemize} 
    }
\end{frame}

\begin{frame}{Relativität von Längen}
    \begin{figure}[h]
        \centering
        \includegraphics[width=0.8\textwidth]{relLaengenB1}
    \end{figure}
    \vspace{-10pt}
    \begin{figure}[h]
        \centering
        \includegraphics[width=0.8\textwidth]{relLaengenB2}
        \caption{\cite{Einstein1905}}
    \end{figure}
    \begin{figure}[h]
        \centering
        \includegraphics[width=0.7\textwidth]{relLaengeBewegtRuhend}
    \end{figure}
\end{frame}

\section{§3 Transformationen}
\begin{frame}{Translation entlang der x-Achse}
    \begin{figure}[h]
        \centering
        \includegraphics[width=0.9\textwidth]{translationX1}
    \end{figure}
\end{frame}

\begin{frame}{Zwei Koordinatensysteme / Raumzeiten}
    \begin{figure}[h]
        \centering
        \includegraphics[width=0.5\textwidth]{translationX1}
    \end{figure}
    \begin{columns}
        \begin{column}{0.5\textwidth}
            \begin{align*}
                \begin{pmatrix}
                    x\\y\\z\\t
                \end{pmatrix}_{K}
            \end{align*}
        \end{column}
        \begin{column}{0.5\textwidth}
            \begin{align*}
                \begin{pmatrix}
                    \xi = \text{xi}\\ 
                    \eta = \text{eta}\\ 
                    \zeta =\text{zeta}\\ 
                    \tau = \text{tau}
                \end{pmatrix}_{k}
            \end{align*}
        \end{column}
    \end{columns}
\end{frame}

\begin{frame}{Ein im bewegten System ruhender Punkt}
    \begin{figure}[h]
        \centering
        \includegraphics[height=0.3\textwidth]{translationX2T0}
        \hspace{30pt}
        \includegraphics[height=0.3\textwidth]{translationX2T}
    \end{figure}
    \begin{equation*}
        x(t)        = v\cdot t + x^\prime \quad \quad| -v \cdot t 
    \end{equation*}
    \begin{equation*}
        \boxed{
            x^\prime    = x(t) - v \cdot t
        }
    \end{equation*}
\end{frame}

\begin{frame}{Alternative Darstellung in K}
    \begin{figure}[h]
        \centering
        \includegraphics[height=0.25\textwidth]{translationX2T}
    \end{figure}
    \begin{equation*}
        \boxed{
            x^\prime    = x(t) - v \cdot t
        }
    \end{equation*}
    \begin{columns}
        \begin{column}{0.1\textwidth}
            \begin{align*}
                \begin{pmatrix}
                    x^\prime\\y\\z\\t
                \end{pmatrix}_{K}
            \end{align*}
        \end{column}
        \begin{column}{0.1\textwidth}
            statt
        \end{column}
        \begin{column}{0.1\textwidth}
            \begin{align*}
                \begin{pmatrix}
                    x\\y\\z\\t
                \end{pmatrix}_{K}
            \end{align*}
        \end{column}
    \end{columns}
\end{frame}

\begin{frame}{Synchrone bewegte Uhren}
    \begin{figure}[h]
        \centering
        \includegraphics[height = 0.3\textwidth]{synchronBewegt}
    \end{figure}
    \vspace{-20pt}
    \begin{figure}[h]
        \centering
        \begin{tikzpicture}
            \tikzmath{
                \xA = -2.5;
                \xB = 4;
                \yStart = -1.5;
                \dY = -0.5;
                \lD = 0.5;
            }
            \draw (\xA,\yStart+\dY) -- ++(0,-1.5);
            \draw (\xB,\yStart+\dY) -- ++(0,-1.5cm);
            % Draw light forward
            \draw [line width = 2pt, ->,red] 
                (\xA,\yStart+\dY*2) 
                node [label={[label distance=\lD,black]left:{$\tau_0$}}]{}
                -- (\xB,\yStart+\dY*2) 
                node [label={[label distance=\lD,black]right:{$\tau_1$}}]{};
            % Draw light backward
            \draw [line width = 2pt, ->,red] 
                (\xB,\yStart+\dY*3) 
                node [label={[label distance=\lD,black]right:{}}]{}
                -- (\xA,\yStart+\dY*3) 
                node [label={[label distance=\lD,black]left:{$\tau_2$}}]{};
        \end{tikzpicture}
    \end{figure}
    \only<2->{
        Synchronisationsbedingung:       
        \begin{align*}
            \tau_2 - \tau_1     &=  \tau_1 + \tau_0             &&| + \tau_0 ; +\tau_1\\
            \tau_2 + \tau_0     &=  2\cdot \tau_1               &&| :2\\
            \Aboxed{
            \tfrac{1}{2}\left(\tau_1 - \tau_0\right)     &=  \tau_1
            }
        \end{align*}
    }
\end{frame}

\begin{frame}{Ziel:}
    \begin{itemize}
        \item Bestimme:
    \end{itemize}
\begin{equation*}
    \begin{pmatrix}
        \xi \\ 
        \eta \\ 
        \zeta\\ 
        \tau 
    \end{pmatrix}
=
    \begin{pmatrix}
        \xi     \left( x^\prime, y, z, t\right) \\ 
        \eta    \left( x^\prime, y, z, t\right) \\  
        \zeta   \left( x^\prime, y, z, t\right) \\  
        \tau    \left( x^\prime, y, z, t\right) \\  
    \end{pmatrix}
\end{equation*}
\end{frame}

\begin{frame}{Ansatz für $\tau$}
    \begin{figure}[h]
        \centering
        \includegraphics[height=0.12\textwidth]{linear}
        \caption{\cite{Einstein1905}}
    \end{figure}
    Daraus folgt für $\tau$:
    \begin{equation*}
        \tau    \left( x^\prime, y, z, t\right)
        =
        \frac{\partial \tau}{\partial x^\prime} x^\prime
        +
        \frac{\partial \tau}{\partial y} y
        +
        \frac{\partial \tau}{\partial z} z
        +
        \frac{\partial \tau}{\partial t} t
    \end{equation*}
\end{frame}

\begin{frame}{Die Zeit $\tau$ im bewegten System}
    \begin{figure}[h]
        \centering
        \includegraphics[height = 0.2\textwidth]{synchronBewegt}\\
        \begin{tikzpicture}[scale=0.5]
            \tikzmath{
                \xA = -2.5;
                \xB = 4;
                \yStart = -1.5;
                \dY = -0.5;
                \lD = 0.5;
            }
            \draw (\xA,\yStart+\dY) -- ++(0,-1.5);
            \draw (\xB,\yStart+\dY) -- ++(0,-1.5cm);
            % Draw light forward
            \draw [line width = 2pt, ->,red] 
                (\xA,\yStart+\dY*2) 
                node [label={[label distance=\lD,black]left:{$\tau_0$}}]{}
                -- (\xB,\yStart+\dY*2) 
                node [label={[label distance=\lD,black]right:{$\tau_1$}}]{};
            % Draw light backward
            \draw [line width = 2pt, ->,red] 
                (\xB,\yStart+\dY*3) 
                node [label={[label distance=\lD,black]right:{}}]{}
                -- (\xA,\yStart+\dY*3) 
                node [label={[label distance=\lD,black]left:{$\tau_2$}}]{};
        \end{tikzpicture}
    \end{figure}
    Synchronisationsbedingung:
    \begin{align*}
        \tfrac{1}{2}\left(\tau_1 - \tau_0\right)     &=  \tau_1\\
        \tfrac{1}{2}\left(
            \tau \left( 0,0,0,t  \right)
            + \tau \left( 0,0,0,t + \tfrac{x^\prime}{V-v} + \tfrac{x^\prime}{V+v} \right)
        \right)
            &=
            \tau \left( x^\prime,0,0,t+\tfrac{x^\prime}{V-v}  \right)
    \end{align*}
\end{frame}

\begin{frame}{Die Zeit $\tau$ im bewegten System}
    Mit dem Ansatz 
    \begin{equation*}
        \tau    \left( x^\prime, y, z, t\right)
        =
        \frac{\partial \tau}{\partial x^\prime} x^\prime
        +
        \frac{\partial \tau}{\partial y} y
        +
        \frac{\partial \tau}{\partial z} z
        +
        \frac{\partial \tau}{\partial t} t
    \end{equation*}
    gilt für Bestandteile der Synchronisationsbedingung:
    \begin{align*}
            \tau \left( 0,0,0,t  \right) 
                &=
                    \frac{\partial \tau}{\partial x^\prime} 0
                    +
                    \frac{\partial \tau}{\partial y} 0
                    +
                    \frac{\partial \tau}{\partial z} 0
                    +
                    \frac{\partial \tau}{\partial t} \textcolor{red}{t}\\
            \tau \left( 0,0,0,t + \tfrac{x^\prime}{V-v} + \tfrac{x^\prime}{V+v} \right)
                &= 
                    \frac{\partial \tau}{\partial x^\prime} 0
                    +
                    \frac{\partial \tau}{\partial y} 0
                    +
                    \frac{\partial \tau}{\partial z} 0
                    +
                    \frac{\partial \tau}{\partial t} 
                        \left(
                            \textcolor{red}{
                                t + \tfrac{x^\prime}{V-v} + \tfrac{x^\prime}{V+v} 
                            }
                        \right)\\
            \tau \left( x^\prime,0,0,t+\tfrac{x^\prime}{V-v}  \right)
                &= 
                    \frac{\partial \tau}{\partial x^\prime} \textcolor{red}{x^\prime}
                    +
                    \frac{\partial \tau}{\partial y} 0
                    +
                    \frac{\partial \tau}{\partial z} 0
                    +
                    \frac{\partial \tau}{\partial t}
                        \left(
                            \textcolor{red}{
                                t+\tfrac{x^\prime}{V-v}
                            }
                        \right)
    \end{align*}
\end{frame}

\begin{frame}
    Synchronisationsbedingung in x-Richtung:
    \begin{align*}
        \tfrac{1}{2}
            \left(
                \tau \left( 0,0,0,t  \right)
                + \tau \left( 0,0,0,t + \tfrac{x^\prime}{V-v} + \tfrac{x^\prime}{V+v} \right)
            \right)
        &=
            \tau \left( x^\prime,0,0,t+\tfrac{x^\prime}{V-v}  \right)\\
        %%%
        \textcolor{green}{\tfrac{1}{2}}\left[
            \frac{\partial \tau}{\partial t} \textcolor{green}{t} 
            +\frac{\partial \tau}{\partial t} \left(
                \textcolor{green}{t} + x^\prime \left( 
                    \tfrac{1}{V-v} + \tfrac{1}{V+v}
                \right)
            \right)
        \right]
        &=
        \frac{\partial \tau}{\partial x^\prime} x^\prime 
        +\frac{\partial \tau}{\partial t}\left(
            \textcolor{green}{t}+\tfrac{x^\prime}{V-v}
        \right)\\
        %%%
        \tfrac{1}{2}
        \frac{\partial \tau}{\partial t} \left(
            \tfrac{1}{V-v} + \tfrac{1}{V+v}
        \right)\textcolor{green}{x^\prime}
        &=
        \left(
            \frac{\partial \tau}{\partial x^\prime} 
            +\frac{\partial \tau}{\partial t}
            \tfrac{1}{V-v}
        \right)\textcolor{green}{x^\prime}\\
        %%%
        \tfrac{1}{2}
        \frac{\partial \tau}{\partial t} \left(
            \tfrac{V+v+V-v}{V^2-v^2}
        \right)
        &=
        \frac{\partial \tau}{\partial x^\prime} 
        +\frac{\partial \tau}{\partial t}
        \tfrac{V+v}{V^2-v^2}
    \end{align*}
    \begin{equation*}
        \boxed{
            \frac{\partial \tau}{\partial x^\prime} 
        }
        =
        -\tfrac{v}{V^2-v^2}
        \boxed{
            \frac{\partial \tau}{\partial t} 
        }
    \end{equation*}
Wir kennen nun den Koeffizienten 
            $\frac{\partial \tau}{\partial x^\prime}$ 
des Ansatzes als Funktion des Koeffizienten
            $\frac{\partial \tau}{\partial t}$
\end{frame}

\begin{frame}
    Aus den Synchronisationsbedingungen in y-Richtung folgt:
    \begin{equation*}
        \boxed{
            \frac{\partial \tau}{\partial y} 
        }
        =
            0
    \end{equation*}
    Aus den Synchronisationsbedingungen in z-Richtung folgt:
    \begin{equation*}
        \boxed{
            \frac{\partial \tau}{\partial z} 
        }
        =
            0
    \end{equation*}
\end{frame}

\begin{frame}
    Wir setzen 
        $\frac{\partial \tau}{\partial t}$
    als Funktion der Geschwindigkeit mit 
        $\frac{\partial \tau}{\partial t} = a(v)$
    an.\\\vspace{20pt}
    Mit
    \begin{equation*}
        \boxed{
            \frac{\partial \tau}{\partial x^\prime} 
        }
        =
        -\tfrac{v}{V^2-v^2}
        \boxed{
            \frac{\partial \tau}{\partial t} 
        }
    \end{equation*}
    folgt für den Ansatz:
    \begin{align*}
        \tau    \left( x^\prime, y, z, t\right)
        &=
        \frac{\partial \tau}{\partial x^\prime} x^\prime
        &&+
        \frac{\partial \tau}{\partial y} y
        &&+
        \frac{\partial \tau}{\partial z} z
        &&+
        \frac{\partial \tau}{\partial t} t\\
        %%%
        &=
        \left( 
            -\tfrac{v}{V^2-v^2}
        \right) a\left( v \right) x^\prime
        &&+
        0 
        &&+
        0 
        &&+
        a\left( v \right) t 
    \end{align*}
    \begin{align*}
        \tau    \left( x^\prime, y, z, t\right)
        =
        a\left( v \right) 
        \left( 
            t-\tfrac{v}{V^2-v^2} x^\prime
        \right)
    \end{align*}
\end{frame}

\section{§4 Bedeutung}
\begin{frame}
\end{frame}

\appendix
\section{appendix}
\begin{frame}
\end{frame}

\begin{frame}{Literatur}
    \bibliographystyle{./my_elsart-harv_engl}
    \bibliography{./bib}
\end{frame}

\end{document}
 
 
 
 
 

% \begin{frame}{Konstanz der Lichtgeschwindigkeit}
%     \begin{columns}
%         \begin{column}{-1.5\textwidth}
%             \begin{figure}[h]
%                 \centering
%                 \includegraphics[width = \textwidth]{konstGeschwRuhend}
%                 \caption{\cite{Einstein1904}}
%             \end{figure}
%         \end{column}
%         \begin{column}{-1.5\textwidth}
%             \begin{figure}[h]
%                 \centering
%                 \includegraphics[width = \textwidth]{konstGeschwBewegt}
%                 \caption{\cite{Einstein1904}}
%             \end{figure}
%         \end{column}
%     \end{columns}
% \end{frame}











